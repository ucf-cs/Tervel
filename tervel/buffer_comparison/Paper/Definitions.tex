% !TEX root = WaitFreeRingBuffer.tex

Structures:
%Warning: Lstlisting does not ignore white space
\begin{itemize}

\item RingBuffer:
\begin{lstlisting}
{atomic array[],  atomic int head,
atomic int tail}
\end{lstlisting}

\item NullNode
\begin{lstlisting}
{const long seqid}
\end{lstlisting}

\item ElemNode
\begin{lstlisting}
{const long seqid, const Element element}
\end{lstlisting}

\item Helper
\begin{lstlisting}
{const Op *op, const long seqid, Node *old}
\end{lstlisting}

\item EnqueueOp
\begin{lstlisting}
{const Element, atomic Helper *helper}
\end{lstlisting}

\item DequeueOp
\begin{lstlisting}
{atomic Helper *helper}
\end{lstlisting}

\end{itemize}

Supporting Functions:
\TODO{Add 1-2 sentence descriptions}
\begin{itemize}
\item isNull():
\item isElement():
\item isSkipped():
\item setSkipped():
\item makeSkipped():
\item isHelper():
\item getNextTail():
\item getNextHead():
\item getPosition():
\item BackOff():
\end{itemize}



\subsection{Progress Assurance}
	\label{sec:definitions:progress_assurance}

For a design to be wait-free, a thread must not be continually denied access to a necessary resource.
This design employs a progress assurance scheme to prevent thread starvation.
Without this progress assurance scheme, the design presented may encounter a rare condition in which threads starve.
With larger ring buffer sizes, this possibility can be reduced further.

The progress assurance scheme is designed similar to the announcement table presented by Herlihy~\cite{herlihy_table}.
Threads will check the table incrementally at the start of every operation and help complete any operation found as presented by Kogan~\cite{kogan_fpsp}.
Our design is inpired by the design presented by Feldman, which is based on these two designs~\cite{feldman_vector}. % yes??
This design uses an announcement table of \emph{OpRec} which contains a \SYN{control word} indicating an operations state.
When the control word is a reference to a descriptor object the operation has been completed, otherwise, threads will continue to help.



\TODO{Describe it, cite Bible's Annoucement Table, Kogans method, Vector's Descriptor Association}

%What I want you to do is read the Progresss Assurance scheme section from my vector.
%Do a 5 sentences or less summary, then present a ring buffer specific implementation.